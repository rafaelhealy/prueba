\documentclass[12pt, a4paper]{article}
\def\spanishoptions{mexico-com}
\usepackage[spanish]{babel}
\usepackage[utf8]{inputenc} 
\usepackage{lmodern}
\usepackage[T1]{fontenc}
\usepackage{textcomp} 
\usepackage{hyperref} 

\title{Instalar \LaTeX}
\author{Rafael Healy Chacón}
\date{Febrero 2013}
\newpage
\begin{document}
\maketitle

\section{Instalación}aS
\subsection{OSX}
\begin{enumerate}
\item Instalar la distribución MacTex de \LaTeX para OSX disponible en: \url{http://tug.org/mactex/}.
\item Instalar Texmaker para OSX disponible en: \url{http://www.xm1math.net/texmaker/download.html}.
\end{enumerate}

\subsection{Windows}
\begin{enumerate}
\item Instalar la distribución MiKTeX de \LaTeX para Windows disponible en: \url{http://miktex.org/download}.
\item Instalar Texmaker para Windows disponible en: \url{http://www.xm1math.net/texmaker/download.html}.
\end{enumerate}

\section{Configurar Texmaker}

El orden de compilación varia dependiendo del documento. Para documentos simples se puede usar PdfLaTeX + Pdf~view. Para documentos con citas y referencias PdfLaTeX + BibTex + PdfLaTeX + PdfLaTeX + Pdf~view.

\subsection{OSX}
Ir a texmaker -> Preferencias... -> Quick Build y seleccionar User e utilizar el wizard.

\subsection{Windows}
Ir a Opciones -> Configurar Texmaker -> Quick Build y seleccionar User e utilizar el wizard.

\section{Git}

Utilizar un sistema de control de versiones Git para sincronizar los archivos (La)tex. El sistema debe ignorar los archivos auxiliares con un .gitignore que incluya las siguientes extensiones:

\begin{itemize}
\item (La)TeX: *.log *.aux *.toc
\item Listas: *.lof *.lot
\item Índices: *.idx *.ind *.ilg
\item Glosarios: *.glo *.gls *.glg
\item BibTeX: *.bbl *.blg
\item hyperref: *.out
\item svn-multi: *.svn *.svt *.svx
\item standalone: *.sta *.stp
\item latexmk tool: *.fdb\_latexmk
\item endfloat: *.fff
\item gzip: *.gz
\item Otros: *.run.xml
\end{itemize}

\end{document}