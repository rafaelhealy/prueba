\documentclass[12pt, a4paper]{article}
\def\spanishoptions{mexico-com}
\usepackage[spanish]{babel}
\usepackage[utf8]{inputenc} 
\usepackage{lmodern} 
\usepackage[T1]{fontenc}  
\usepackage{hyperref} 

\title{Instalar \LaTeX}
\author{Rafael Healy Chacón}
\date{Febrero 2013}
\newpage
\begin{document}
\maketitle

\section{Instalación}
\subsection{OSX}
\begin{enumerate}
\item Instalar la distribución MacTex de \LaTeX para OSX disponible en: \url{http://tug.org/mactex/}.
\item Instalar Texmaker para OSX disponible en: \url{http://www.xm1math.net/texmaker/download.html}.
\end{enumerate}

\subsection{Windows}
\begin{enumerate}
\item Instalar la distribución MiKTeX de \LaTeX para Windows disponible en: \url{http://miktex.org/download}.
\item Instalar Texmaker para Windows disponible en: \url{http://www.xm1math.net/texmaker/download.html}.
\end{enumerate}

\section{Configurar Texmaker}
\paragraph{Quick Build}
En opciones de quickbuild usar el wizzard con: PdfLaTeX + BibTex + PdfLaTeX + PdfLaTeX + Pdf view(opcional).

\end{document}