%%%%%%%%%%%%%%%% PAQUETES %%%%%%%%%%%%%%%%
\documentclass[12pt, a4paper,twoside]{article} %Tamaño de letra, papel e impresión en 2 caras
\def\spanishoptions{mexico-com}
\usepackage[spanish]{babel}
\usepackage[utf8]{inputenc} %Para poder escribir acentos directiamente
\usepackage{lmodern} %Mejor que el tipo de letra por defecto
\usepackage[T1]{fontenc} %Para escribir algunos caracteres especiales y que copiar/pegar palabras con acento desde los pdfs funcione bien.
\usepackage{textcomp} %Para usar otros caracteres especiales como: grados "°" (grados) que es diferente que "º" y euros "€".
\usepackage{microtype} %Para expandir o comprimir el texto justificado y que se vea mejor
\usepackage[left=25mm, top=25mm, bottom=25mm, right=25mm]{geometry} %Márgenes de la hoja
\usepackage[section]{placeins} %evita que floats de una sección aparescan en otra
\usepackage{pdflscape} %Para tener páginas horizontales dentro del documento
\usepackage{appendix} %Apéncices
\usepackage{hyperref} %Vínculos
\usepackage{amsmath} %Fórmulas
\usepackage{graphicx} %Imágenes
\usepackage{subcaption} %Para colocar varias imágenes como una sola figura
%\usepackage{tabu}
%\usepackage{tabulary}
\usepackage[table]{xcolor}
\usepackage{listings} %Escribir código
% \usepackage{gensymb} %Para escribir grados

%%%%\def\input@path{{/path/to/folder}}

\lstset{language=C} %Lenguage del código
\usepackage[space]{grffile} %Permite tener espacios en las rutas de archivos
\graphicspath{{/Users/rafa/Dropbox/imagenes/}{C:/Users/Rafael Healy/Pictures/Imagenes/}} %Ubicación de las imágenes del documento
%\setlength{\parindent}{1cm} %Tamaño de la sangría
%\usepackage{setspace} %Intelineado
	%\singlespacing
	%\onehalfspacing
	%\doublespacing
	%\setstretch{1.1}

%Renobrar a español
%\renewcommand{\contentsname}{Contenido}
%\renewcommand{\listfigurename}{Índice de figuras}
%\renewcommand{\listtablename}{Índice de tablas}
%\renewcommand\refname{Bibliografía}
\renewcommand{\appendixname}{Apéndices}
\renewcommand{\appendixtocname}{Apéndices}
\renewcommand{\appendixpagename}{Apéndices}

\title{Documento para probar \LaTeX}
\author{Rafael Healy Chacón}
\date{Febrero 2013}
\newpage
\begin{document}
\maketitle
\clearpage

\tableofcontents
\addtocontents{toc}{~\hfill\textbf{Página}\par}
\listoffigures
\listoftables

\clearpage
\section{Texto}
Salto de linea con linea vacía. Empieza con sangría por defecto.\\
Negritas: \textbf{negritas}\\
Énfasis: \emph{énfasis}\\
Itálicas: \textit{Es lo mismo que énfasis pero al usar énfasis \emph{dentro} de itálicas pone el texto normal para darle énfasis.}

\subsection{Ecuaciones}
Notación matemática in-line: $(x+1)$ y $(x+3)$.

Notación matemática separada: $$3x^{4e} - A_b$$ 

Ecuaciones numeradas: 
\begin{equation} 
x=2y^3 
\end{equation}

Varias alineadas:
\begin{align}
u &= \arctan x & dv &= 1 \, dx
\\ du &= \frac{1}{1 + x^2} dx & v &= x.
\end{align}

Con etiqueta para hacerle referencia en el texto
\begin{equation} \label{eq:miec}
5^2 - 5 = 20
\end{equation}

Referencia general a una ecuación: \ref{eq:miec}. 

Referencia de ecuación: \eqref{eq:miec}.

\subsection{Vínculos}
Simple: \url{google.com}

Con texto: \href{google.com}{Google}

Una cita de ejemplo: \cite{texbook}

\begin{center}
centrado %comentario 
\end{center}

Para hacer que dos palabras espaciadas no se separen por salto de linea separarlas por un \textasciitilde{}. Por ejemplo al escribir:  Pag.~45.

Código:
\begin{lstlisting}
# include <iostream>
 
int main()
{
	std::cout << "Hello, world!\n";
}
\end{lstlisting}

\section{Listas}
\subsection{Lista numerada}
\begin{enumerate}
\item perro
\item gato
\item ratón
	\begin{enumerate}
	\item Jerry
	\item Speedy
	\end{enumerate}
\end{enumerate}
\subsection{Lista con viñetas}
\begin{itemize}
\item perro
\item gato
\item ratón
	\begin{itemize}
	\item Jerry
	\item Speedy
	\end{itemize}
\end{itemize}

\section{Imágenes}
Incluir el paquetes \emph{graphicx}. Para crear figuras de subfiguras usar \emph{subcaption}. 

Para imprimir el documento sin las imágenes y solo con cuadros vacíos colocar el parámetro \emph{draft} en \emph{documentclass}. 

Para utilizar un repositorio común de imágenes entre colaboradores sincronizar las imágenes con algo como Dropbox y con el comando \emph{\textbackslash graphicspath} especificar la ruta de la carpeta de imágenes de cada colaborador.

\begin{figure}[!ht]
	\centering
		\includegraphics[width=0.5\textwidth]{oso}
		\caption[Foto de un oso.]{Foto de un oso, el 					animal más grande del bosque.}
		\label{fig:oso}
\end{figure}

Una referencia a la figura \ref{fig:oso}.

\begin{figure}
	\centering
	\begin{subfigure}[b]{0.3\textwidth}
		\centering
		\includegraphics[width=\textwidth]{oso}
		\caption{un oso}
		\label{subfig:oso}
	\end{subfigure}
	\quad
	\begin{subfigure}[b]{0.3\textwidth}
		\centering
		\includegraphics[width=\textwidth]{tigre}
		\caption{Un tigre}
		\label{subfig:tigre}
	\end{subfigure}
	\begin{subfigure}[b]{0.3\textwidth}
		\centering
		\includegraphics[width=\textwidth]{perro}
		\caption{Un perro}
		\label{subfig:perro}
	\end{subfigure}
	\caption{Fotos de animales}\label{fig:animals}
\end{figure}

\section{Tablas}
\subsection{Tabla simple}
\begin{tabular}{| l | c | r |}
	\hline
  1 & 2 & 3 \\ \hline
  4 & 5 & 6 \\ \hline
  7 & 8 & 9 \\ \hline
\end{tabular}

\subsection{Tabla con text wrap}
\begin{center}
    \begin{tabular}{ | l | l | l | p{6cm} |}
    \hline
    Day & Min Temp & Max Temp & Summary \\ \hline
    Monday & 11C & 22C & Lorem ipsum ad his scripta blandit partiendo, eum fastidii accumsan euripidis in, eum liber hendrerit an.\\ \hline
    Tuesday & 9C & 19C & Lorem ipsum ad his scripta blandit partiendo, eum fastidii accumsan euripidis in, eum liber hendrerit an. \\ \hline
    Wednesday & 10C & 21C & Lorem ipsum ad his scripta blandit partiendo, eum fastidii accumsan euripidis in, eum liber hendrerit an. \\
    \hline
    \end{tabular}
\end{center}

\subsection{Tabla para hacerle referencia en texto y que aparezca en la TOC}
\begin{table}[!ht]
  \centering
  \begin{tabular}{|l|}
  		\hline
  		la tabla \\
  		\hline
  \end{tabular}
  \caption{Una tabla con epígrafe}
  \label{tab:tablaEjemplo}
\end{table}

\subsection{Tabla desde OpenOffice}
\begin{enumerate}
\item Instalar Calc2LaTeX en OpenOffice.
\item Hacer la tabla en una hoja de cálculo y seleccionarla.
\item Ir a Herramientas -> Macros -> Ejecutar macro...
\item Expandir Mis macros y ejecutar el macro Main de Calc2LaTeX.
\item (Opcional) Añadir un botón del macro a una barra de herramientas.
\begin{enumerate}
\item Herramientas -> Personalizar... -> Barras de herramientas.
\item Seleccionar la barra a la que se quiere agregar el botón y seleccionar Agregar...
\item Buscar el macro Main en Macros de OpenOffice.org y seleccionar Agregar,
\item (Opcional) Modificar el botón con el nombre Calc2LaTeX y poner un icono.
\end{enumerate}

\end{enumerate}

\begin{table}[htbp]
\begin{center}
\begin{tabular}{|l|l|c|c|}
\hline
\textbf{Pais} & \textbf{Capital} & \multicolumn{ 2}{c|}{\textbf{Moneda}} \\ \hline
México & D.F. & \$ & Peso \\ \hline
Alemania & Berlín & \multicolumn{ 1}{c|}{€} & \multicolumn{ 1}{c|}{Euro} \\ \cline{ 1- 2}
Francia & Paris & \multicolumn{ 1}{c|}{} & \multicolumn{ 1}{c|}{} \\ \hline
\end{tabular}
\end{center}
\caption{Ejemplo de tabla hecha en OpenOffice}
\label{tab:openoffice}
\end{table}

\begin{landscape}
\section{Página horizontal}
Texto
\end{landscape}

\section{Citas y bibliografía}
\clearpage
\bibliographystyle{plain}
\bibliography{/Users/rafa/Documents/Prueba/referencias}

\clearpage
\appendix
\addappheadtotoc
\appendixpage
\section{Primer apéndice}

\end{document}


