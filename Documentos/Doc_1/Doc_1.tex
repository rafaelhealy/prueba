\documentclass[12pt]{article}
\usepackage[utf8]{inputenc}
\usepackage{hyperref}
\usepackage{amsmath}
\title{Plantilla de Latex}
\author{Pedro Iván Romero Ojeda}
\date{Febrero 2013}
\begin{document}
\maketitle
Mi primer texto. Notación matemática in-line $(x+1)$ y $(x+3)$ Notación matemática separada con doble signo: $$3x^{4e} - A_b$$

Para más información de fórmulas y expresiones matemáticas click \href{http://en.wikibooks.org/wiki/LaTeX/Mathematics}{aquí} y para como hacer hipervínculos \href{http://en.wikibooks.org/wiki/LaTeX/Hyperlinks}{aquí}

Salto de linea con linea vacía. Empieza con sangría por defecto.
\begin{center}
centrado %comentario 
\end{center}


adskfjbasldjfbnasñdjfnaosdjnf adfjgnaopand opgajnsdfpogasfgn apodg asodg naosd na po gnaop gnasdg p edito el primer perrafo!!!!!!!!! aaaaaaaaaaaaaaaaaaa

a i aipdgfuaopsdf asodfgijsadògijsd`fog ijsd ogisdj gposdfijgspdo figjs dpofgi jsdfpog ijsdpgo sdi gjsdpofg ijsdf\\
XXXXXXdñfgsfgo dfgopisdfgopsdjifgopsdjnfgsodpfjgnpsdofgnsdfo

adskfjbasldjfbnasñdjfnaosdjnf adfjgnaopand opgajnsdfpogasfgn apodg asodg naosd na po gnaop gnasdg p

a i aipdgfuaopsdf asodfgijsadògijsd`fog ijsd ogisdj gposdfijgspdo figjs dpofgi jsdfpog ijsdpgo sdi gjsdpofg ijsdf\\
XXXXXXdñfgsfgo dfgopisdfgopsdjifgopsdjnfgsodpfjgnpsdofgnsdfoadskfjbasldjfbnasñdjfnaosdjnf adfjgnaopand opgajnsdfpogasfgn apodg asodg naosd na po gnaop gnasdg p

a i aipdgfuaopsdf asodfgijsadògijsd`fog ijsd ogisdj gposdfijgspdo figjs dpofgi jsdfpog ijsdpgo sdi gjsdpofg ijsdf\\
XXXXXXdñfgsfgo dfgopisdfgopsdjifgopsdjnfgsodpfjgnpsdofgnsdfoadskfjbasldjfbnasñdjfnaosdjnf adfjgnaopand opgajnsdfpogasfgn apodg asodg naosd na po gnaop gnasdg p

a i aipdgfuaopsdf asodfgijsadògijsd`fog ijsd ogisdj gposdfijgspdo figjs dpofgi jsdfpog ijsdpgo sdi gjsdpofg ijsdf\\
XXXXXXdñfgsfgo dfgopisdfgopsdjifgopsdjnfgsodpfjgnpsdofgnsdfo

\begin{align}
m &= \frac{m_0}{\sqrt{1-\frac{v^2}{c^2}}}
\end{align}



\end{document}
